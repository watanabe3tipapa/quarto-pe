% Options for packages loaded elsewhere
\PassOptionsToPackage{unicode}{hyperref}
\PassOptionsToPackage{hyphens}{url}
\PassOptionsToPackage{dvipsnames,svgnames,x11names}{xcolor}
%
\documentclass[
  a4paper,
]{article}

\usepackage{amsmath,amssymb}
\usepackage{iftex}
\ifPDFTeX
  \usepackage[T1]{fontenc}
  \usepackage[utf8]{inputenc}
  \usepackage{textcomp} % provide euro and other symbols
\else % if luatex or xetex
  \usepackage{unicode-math}
  \defaultfontfeatures{Scale=MatchLowercase}
  \defaultfontfeatures[\rmfamily]{Ligatures=TeX,Scale=1}
\fi
\usepackage{lmodern}
\ifPDFTeX\else  
    % xetex/luatex font selection
\fi
% Use upquote if available, for straight quotes in verbatim environments
\IfFileExists{upquote.sty}{\usepackage{upquote}}{}
\IfFileExists{microtype.sty}{% use microtype if available
  \usepackage[]{microtype}
  \UseMicrotypeSet[protrusion]{basicmath} % disable protrusion for tt fonts
}{}
\makeatletter
\@ifundefined{KOMAClassName}{% if non-KOMA class
  \IfFileExists{parskip.sty}{%
    \usepackage{parskip}
  }{% else
    \setlength{\parindent}{0pt}
    \setlength{\parskip}{6pt plus 2pt minus 1pt}}
}{% if KOMA class
  \KOMAoptions{parskip=half}}
\makeatother
\usepackage{xcolor}
\usepackage[lmargin=30mm,rmargin=30mm]{geometry}
\setlength{\emergencystretch}{3em} % prevent overfull lines
\setcounter{secnumdepth}{5}
% Make \paragraph and \subparagraph free-standing
\ifx\paragraph\undefined\else
  \let\oldparagraph\paragraph
  \renewcommand{\paragraph}[1]{\oldparagraph{#1}\mbox{}}
\fi
\ifx\subparagraph\undefined\else
  \let\oldsubparagraph\subparagraph
  \renewcommand{\subparagraph}[1]{\oldsubparagraph{#1}\mbox{}}
\fi

\usepackage{color}
\usepackage{fancyvrb}
\newcommand{\VerbBar}{|}
\newcommand{\VERB}{\Verb[commandchars=\\\{\}]}
\DefineVerbatimEnvironment{Highlighting}{Verbatim}{commandchars=\\\{\}}
% Add ',fontsize=\small' for more characters per line
\newenvironment{Shaded}{}{}
\newcommand{\AlertTok}[1]{\textcolor[rgb]{1.00,0.33,0.33}{\textbf{#1}}}
\newcommand{\AnnotationTok}[1]{\textcolor[rgb]{0.42,0.45,0.49}{#1}}
\newcommand{\AttributeTok}[1]{\textcolor[rgb]{0.84,0.23,0.29}{#1}}
\newcommand{\BaseNTok}[1]{\textcolor[rgb]{0.00,0.36,0.77}{#1}}
\newcommand{\BuiltInTok}[1]{\textcolor[rgb]{0.84,0.23,0.29}{#1}}
\newcommand{\CharTok}[1]{\textcolor[rgb]{0.01,0.18,0.38}{#1}}
\newcommand{\CommentTok}[1]{\textcolor[rgb]{0.42,0.45,0.49}{#1}}
\newcommand{\CommentVarTok}[1]{\textcolor[rgb]{0.42,0.45,0.49}{#1}}
\newcommand{\ConstantTok}[1]{\textcolor[rgb]{0.00,0.36,0.77}{#1}}
\newcommand{\ControlFlowTok}[1]{\textcolor[rgb]{0.84,0.23,0.29}{#1}}
\newcommand{\DataTypeTok}[1]{\textcolor[rgb]{0.84,0.23,0.29}{#1}}
\newcommand{\DecValTok}[1]{\textcolor[rgb]{0.00,0.36,0.77}{#1}}
\newcommand{\DocumentationTok}[1]{\textcolor[rgb]{0.42,0.45,0.49}{#1}}
\newcommand{\ErrorTok}[1]{\textcolor[rgb]{1.00,0.33,0.33}{\underline{#1}}}
\newcommand{\ExtensionTok}[1]{\textcolor[rgb]{0.84,0.23,0.29}{\textbf{#1}}}
\newcommand{\FloatTok}[1]{\textcolor[rgb]{0.00,0.36,0.77}{#1}}
\newcommand{\FunctionTok}[1]{\textcolor[rgb]{0.44,0.26,0.76}{#1}}
\newcommand{\ImportTok}[1]{\textcolor[rgb]{0.01,0.18,0.38}{#1}}
\newcommand{\InformationTok}[1]{\textcolor[rgb]{0.42,0.45,0.49}{#1}}
\newcommand{\KeywordTok}[1]{\textcolor[rgb]{0.84,0.23,0.29}{#1}}
\newcommand{\NormalTok}[1]{\textcolor[rgb]{0.14,0.16,0.18}{#1}}
\newcommand{\OperatorTok}[1]{\textcolor[rgb]{0.14,0.16,0.18}{#1}}
\newcommand{\OtherTok}[1]{\textcolor[rgb]{0.44,0.26,0.76}{#1}}
\newcommand{\PreprocessorTok}[1]{\textcolor[rgb]{0.84,0.23,0.29}{#1}}
\newcommand{\RegionMarkerTok}[1]{\textcolor[rgb]{0.42,0.45,0.49}{#1}}
\newcommand{\SpecialCharTok}[1]{\textcolor[rgb]{0.00,0.36,0.77}{#1}}
\newcommand{\SpecialStringTok}[1]{\textcolor[rgb]{0.01,0.18,0.38}{#1}}
\newcommand{\StringTok}[1]{\textcolor[rgb]{0.01,0.18,0.38}{#1}}
\newcommand{\VariableTok}[1]{\textcolor[rgb]{0.89,0.38,0.04}{#1}}
\newcommand{\VerbatimStringTok}[1]{\textcolor[rgb]{0.01,0.18,0.38}{#1}}
\newcommand{\WarningTok}[1]{\textcolor[rgb]{1.00,0.33,0.33}{#1}}

\providecommand{\tightlist}{%
  \setlength{\itemsep}{0pt}\setlength{\parskip}{0pt}}\usepackage{longtable,booktabs,array}
\usepackage{calc} % for calculating minipage widths
% Correct order of tables after \paragraph or \subparagraph
\usepackage{etoolbox}
\makeatletter
\patchcmd\longtable{\par}{\if@noskipsec\mbox{}\fi\par}{}{}
\makeatother
% Allow footnotes in longtable head/foot
\IfFileExists{footnotehyper.sty}{\usepackage{footnotehyper}}{\usepackage{footnote}}
\makesavenoteenv{longtable}
\usepackage{graphicx}
\makeatletter
\def\maxwidth{\ifdim\Gin@nat@width>\linewidth\linewidth\else\Gin@nat@width\fi}
\def\maxheight{\ifdim\Gin@nat@height>\textheight\textheight\else\Gin@nat@height\fi}
\makeatother
% Scale images if necessary, so that they will not overflow the page
% margins by default, and it is still possible to overwrite the defaults
% using explicit options in \includegraphics[width, height, ...]{}
\setkeys{Gin}{width=\maxwidth,height=\maxheight,keepaspectratio}
% Set default figure placement to htbp
\makeatletter
\def\fps@figure{htbp}
\makeatother

\makeatletter
\makeatother
\makeatletter
\makeatother
\makeatletter
\@ifpackageloaded{caption}{}{\usepackage{caption}}
\AtBeginDocument{%
\ifdefined\contentsname
  \renewcommand*\contentsname{Table of contents}
\else
  \newcommand\contentsname{Table of contents}
\fi
\ifdefined\listfigurename
  \renewcommand*\listfigurename{List of Figures}
\else
  \newcommand\listfigurename{List of Figures}
\fi
\ifdefined\listtablename
  \renewcommand*\listtablename{List of Tables}
\else
  \newcommand\listtablename{List of Tables}
\fi
\ifdefined\figurename
  \renewcommand*\figurename{Figure}
\else
  \newcommand\figurename{Figure}
\fi
\ifdefined\tablename
  \renewcommand*\tablename{Table}
\else
  \newcommand\tablename{Table}
\fi
}
\@ifpackageloaded{float}{}{\usepackage{float}}
\floatstyle{ruled}
\@ifundefined{c@chapter}{\newfloat{codelisting}{h}{lop}}{\newfloat{codelisting}{h}{lop}[chapter]}
\floatname{codelisting}{Listing}
\newcommand*\listoflistings{\listof{codelisting}{List of Listings}}
\makeatother
\makeatletter
\@ifpackageloaded{caption}{}{\usepackage{caption}}
\@ifpackageloaded{subcaption}{}{\usepackage{subcaption}}
\makeatother
\makeatletter
\@ifpackageloaded{tcolorbox}{}{\usepackage[skins,breakable]{tcolorbox}}
\makeatother
\makeatletter
\@ifundefined{shadecolor}{\definecolor{shadecolor}{rgb}{.97, .97, .97}}
\makeatother
\makeatletter
\makeatother
\makeatletter
\makeatother
\ifLuaTeX
\usepackage[bidi=basic]{babel}
\else
\usepackage[bidi=default]{babel}
\fi
% get rid of language-specific shorthands (see #6817):
\let\LanguageShortHands\languageshorthands
\def\languageshorthands#1{}
\ifLuaTeX
  \usepackage{selnolig}  % disable illegal ligatures
\fi
\IfFileExists{bookmark.sty}{\usepackage{bookmark}}{\usepackage{hyperref}}
\IfFileExists{xurl.sty}{\usepackage{xurl}}{} % add URL line breaks if available
\urlstyle{same} % disable monospaced font for URLs
\hypersetup{
  pdftitle={Content1},
  pdfauthor={watanabe3tipapa},
  pdflang={jp},
  colorlinks=true,
  linkcolor={blue},
  filecolor={Maroon},
  citecolor={Blue},
  urlcolor={Blue},
  pdfcreator={LaTeX via pandoc}}

\title{Content1}
\usepackage{etoolbox}
\makeatletter
\providecommand{\subtitle}[1]{% add subtitle to \maketitle
  \apptocmd{\@title}{\par {\large #1 \par}}{}{}
}
\makeatother
\subtitle{An Example of Best Practices}
\author{watanabe3tipapa}
\date{2024-11-21}

\begin{document}
\maketitle
\ifdefined\Shaded\renewenvironment{Shaded}{\begin{tcolorbox}[borderline west={3pt}{0pt}{shadecolor}, interior hidden, boxrule=0pt, enhanced, frame hidden, breakable, sharp corners]}{\end{tcolorbox}}\fi

\renewcommand*\contentsname{Table of contents}
{
\hypersetup{linkcolor=}
\setcounter{tocdepth}{3}
\tableofcontents
}
\hypertarget{my-quarto-project}{%
\section{My Quarto Project}\label{my-quarto-project}}

このウェブサイトで使用しているYAML

\begin{center}\rule{0.5\linewidth}{0.5pt}\end{center}

\hypertarget{yaml-in-building-this-website}{%
\subsubsection{YAML in building this
website}\label{yaml-in-building-this-website}}

\begin{Shaded}
\begin{Highlighting}[]
\FunctionTok{project}\KeywordTok{:}
\AttributeTok{  }\FunctionTok{type}\KeywordTok{:}\AttributeTok{ website}
\AttributeTok{  }\FunctionTok{output{-}dir}\KeywordTok{:}\AttributeTok{ \_docs}
\AttributeTok{  }\FunctionTok{render}\KeywordTok{:}
\AttributeTok{    }\KeywordTok{{-}}\AttributeTok{ }\StringTok{"*.qmd"}
\AttributeTok{    }\KeywordTok{{-}}\AttributeTok{ }\StringTok{"!drafts/"}

\FunctionTok{website}\KeywordTok{:}
\AttributeTok{  }\FunctionTok{title}\KeywordTok{:}\AttributeTok{ }\StringTok{"quarto{-}pe"}
\AttributeTok{  }\FunctionTok{navbar}\KeywordTok{:}
\AttributeTok{    }\FunctionTok{aria{-}label}\KeywordTok{:}\AttributeTok{ Main navigation}
\AttributeTok{    }\FunctionTok{background}\KeywordTok{:}\AttributeTok{ dark}
\AttributeTok{    }\FunctionTok{search}\KeywordTok{:}\AttributeTok{ }\CharTok{true}
\AttributeTok{    }\FunctionTok{left}\KeywordTok{:}
\AttributeTok{      }\KeywordTok{{-}}\AttributeTok{ }\FunctionTok{href}\KeywordTok{:}\AttributeTok{ index.qmd}
\AttributeTok{        }\FunctionTok{text}\KeywordTok{:}\AttributeTok{ Home}
\AttributeTok{      }\KeywordTok{{-}}\AttributeTok{ }\FunctionTok{text}\KeywordTok{:}\AttributeTok{ Contents}
\AttributeTok{        }\FunctionTok{menu}\KeywordTok{:}
\AttributeTok{          }\KeywordTok{{-}}\AttributeTok{ }\FunctionTok{href}\KeywordTok{:}\AttributeTok{ content1.html}
\AttributeTok{            }\FunctionTok{text}\KeywordTok{:}\AttributeTok{ Content1}
\AttributeTok{          }\KeywordTok{{-}}\AttributeTok{ }\FunctionTok{href}\KeywordTok{:}\AttributeTok{ content2.html}
\AttributeTok{            }\FunctionTok{text}\KeywordTok{:}\AttributeTok{ Content2}
\AttributeTok{          }\KeywordTok{{-}}\AttributeTok{ }\FunctionTok{href}\KeywordTok{:}\AttributeTok{ content3.html}
\AttributeTok{            }\FunctionTok{text}\KeywordTok{:}\AttributeTok{ Content3}

\FunctionTok{format}\KeywordTok{:}
\AttributeTok{  }\FunctionTok{html}\KeywordTok{:}
\AttributeTok{    }\FunctionTok{theme}\KeywordTok{:}\AttributeTok{ cosmo}
\AttributeTok{    }\FunctionTok{css}\KeywordTok{:}\AttributeTok{ styles.css}
\AttributeTok{    }\FunctionTok{toc}\KeywordTok{:}\AttributeTok{ }\CharTok{true}
\AttributeTok{    }\FunctionTok{code{-}fold}\KeywordTok{:}\AttributeTok{ }\CharTok{true}
\AttributeTok{    }\FunctionTok{code{-}tools}\KeywordTok{:}\AttributeTok{ }\CharTok{true}
\AttributeTok{  }\FunctionTok{pdf}\KeywordTok{:}
\AttributeTok{    }\FunctionTok{documentclass}\KeywordTok{:}\AttributeTok{ article}
\AttributeTok{    }\FunctionTok{margin{-}left}\KeywordTok{:}\AttributeTok{ 30mm}
\AttributeTok{    }\FunctionTok{margin{-}right}\KeywordTok{:}\AttributeTok{ 30mm}

\FunctionTok{execute}\KeywordTok{:}
\AttributeTok{  }\FunctionTok{echo}\KeywordTok{:}\AttributeTok{ }\CharTok{true}
\AttributeTok{  }\FunctionTok{warning}\KeywordTok{:}\AttributeTok{ }\CharTok{false}
\AttributeTok{  }\FunctionTok{error}\KeywordTok{:}\AttributeTok{ }\CharTok{false}
\AttributeTok{  }\FunctionTok{cache}\KeywordTok{:}\AttributeTok{ }\CharTok{true}

\FunctionTok{freeze}\KeywordTok{:}\AttributeTok{ auto}

\FunctionTok{resources}\KeywordTok{:}
\AttributeTok{  }\KeywordTok{{-}}\AttributeTok{ CNAME}
\AttributeTok{  }\KeywordTok{{-}}\AttributeTok{ .nojekyll}

\FunctionTok{editor}\KeywordTok{:}\AttributeTok{ visual}
\end{Highlighting}
\end{Shaded}

\begin{center}\rule{0.5\linewidth}{0.5pt}\end{center}

\textbf{code 出力}

\textbf{PDF出力(}このページはサイズ(A4)に対応しています\textbf{)}

以上をお試しください。

\begin{center}\rule{0.5\linewidth}{0.5pt}\end{center}

\textbf{YAML に} \{ \texttt{editor:\ visual}
\}\textbf{を追記するといわゆるビジュアル・モードで編集できるようになります。}

\begin{figure}

{\centering \includegraphics{assets/editing_screen.jpg}

}

\caption{編集画面}

\end{figure}

\begin{center}\rule{0.5\linewidth}{0.5pt}\end{center}

\^{}C



\end{document}
